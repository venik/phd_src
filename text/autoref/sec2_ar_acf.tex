\paragraph{Вторая глава} содержит разделы о применении методов параметрического спектрального оценивания и автокоррелляционном анализе в приложении к оценке
параметров широкополосного сигнала. В данной главе приводится обоснование эффективности применения АР модели для оценки параметров широкополосного сигнала.

Особенностью широкополосного сигнала является наличие ярко выраженного одиночного пика в спектральной области сигнала после повторной модуляции ПСП. 
Этот факт дает основания полагать, что такие сигналы могут быть достаточно точно представлены с помощью АР модели второго порядка.
Использование процедуры оценки параметров АР модели вместо перебора по заранее заданным значениям частоты позволяет вести поиск сигнала
в широком диапазоне частот и определять сдвиг с более высокой точностью, устраняя необходимость дополнительного уточнения
доплеровского сдвига перед запуском процедуры сопровождения сигнала.

Точность оценок, полученных на основе АР модели быстро снижается при наличии сильного или окрашенного шума. Для преодоления указанных
трудностей в данной работе предлагается использовать усовершенствованную процедуру многократной переоценки АКФ. 
