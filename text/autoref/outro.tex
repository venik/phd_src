
{\bf{В заключении}}
отражены результаты работы и обозначены направления дальнейшего исследования

\noindent\centerline{\bf{Основные результаты и выводы}}
В ходе диссертационного исследования получены следующие результаты:
\begin{enumerate}
\item Разработан алгоритм на основе параметрического метода оценки частоты для одного источника с широкополосным сигналом.
\item Усовершенствован алгоритм итеративного вычисления автокорреляционной функции, что позволяет использовать его в приемниках
	реального времени.
\item Разработан алгоритм оценки параметров широкополосного сигнала на основе алгоритма Delay and Multiply Approach с использованием
	предложенного усовершенствованного итеративного алгоритма вычисления автокорреляционной функции и параметрического
	метода оценки частоты. Данное решение имеет более высокую точность оценки в сравнении с традиционным
	параллельным коррелятором, в то же время оценка параметра может быть получена за меньшее количество итераций.
\item Произведено имитационное моделирование предложенного алгоритма для проверки положений, выносимых на защиту.
\item Произведено обоснование актуальности и возможности применения параметрического метода оценки частоты для сигналов
	с расширенным спектром.
\item Отражены возможные направления дальнейших исследований в области применения параметрического анализа в системах
	с расширенным спектром.
\end{enumerate}

\paragraph{Основные результаты диссертации отражены в работах:}
\begin{enumerate}
	{\bf{
	\item Шахтарин Б.И., Сидоркина Ю.А., Никифоров А.А. Алгоритм оценки параметров широкополосного сигнала на ограниченном интервале наблюдения //
		Научный вестник МГТУ ГА. 2014. №209 (в публикации)
	\item Никифоров А.А. Оптимизация алгоритма последовательного вычисления автокорреляционной функции //
		Образование. Наука. Научные кадры. 2013 №5. С. 204-207.
	\item Никифоров А.А. Алгоритм итеративного вычисления автокорреляционной функции в задаче оценки частоты широкополосного сигнала //
		Механизация строительства. 2013. № 11 (833). С. 53-55.
	\item Никифоров А.А. Мельников А.О. Токарев С.В. "Детектирование сигналов с расширенным спектром на основе АР модели,
		Промышленные АСУ и контроллеры", 2013. №5 2013, С. 51-54.
	}}

	\item Никифоров А.А., Применение алгоритма Delay and Multiply Approach и АР модели для обнаружения и оценки параметров ШПС //
		доклады 7-ой Всероссийской конференции «Радиолокация и радиосвязь» в рамках Московской Микроволновой Недели. 2013. С. 223-227 
	\item Никифоров А.А. Детектирование сигналов с расширенным спектром на основе АР модели с учетом мощности шума // Доклады международной конференции
		«Радиоэлектронные устройства и системы для инфокоммуникационных технологий - РЕС-2013», М. 2013. Выпуск LXVIII. С. 139-143
	\item Никифоров А.А., Мельников А.О. Методы детектирования систем спутниковой навигации // Интеллектуальный потенциал XXI века:
		ступени познания: Сборник мат-в V Международной студенческой научно-практической конференции. 2011 Часть 2. С. 67 - 70. (0,2 п.л.)

	\item Никифоров А.А. Создание лабораторного стенда для приема сигналов спутниковых систем навигации // ВЕСТНИК Молодых ученых Московского
		государственного университета приборостроения и информатики. М. 2011. №9. C. 55-66
\end{enumerate}
