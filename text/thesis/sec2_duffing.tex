\section{Использование осциллятора Дуффинга для детектирования сигнала}
Теория колебаний и волн возникла примерно в 18 веке. Ее началом принято считать труды Лагранжа, опубликованные в 1788 г. Введя обобщенные
координаты и импульсы, Лагранж, отошел от традиционной механики и записал динамические уравнения, которые могут быть отнесены к системам
любой природы. Подробную историю возникновения теории колебаний и волн можно найти во введении к \cite{landa_book}. 
Введем некоторые используемые в работе термины.

\emph{Аттрактором} называется множество точек в фазовом пространстве, к которому стремятся со
временем все соседние фазовые траектории из некоторой области, называемой областью притяжения \cite{landa_book}.

\emph{Ляпуновский показатель} - 
характеризует степень расходимости близких фазовых траекторий, и число положительных ляпуновских показателей, характеризующее число направлений
неустойчивости. Максимальный ляпуновский показатель:
\begin{center}
\begin{equation}
	\label{eq:exp_lyapunova_1}
	\lambda = \lim \limits_{t \to \infty, d \to 0} = \ln \frac{d(t)}{d(0)},
\end{equation}
\end{center}
где ${d(t)}$ - расстояние между двумя близкими фазовыми траекториями. Непосредственный рассчет показателей по данной формуле является
затруднительным для систем с экспененциальной неустойчивостью траектории. В \cite{landa_book} рассмотрен более простой способ:
\begin{center}
\begin{equation}
	\label{eq:exp_lyapunova_2}
	\lambda = \frac{1}{m}\sum \limits_{i=1}^m \lambda_i = \frac{1}{m\tau}\ln\prod \limits_{i=1}^md_i,
\end{equation}
\end{center}
где локальный ляпуновский показатель ${\lambda_i}=(1/ \tau)\ln d_i$, ${d_i}$ - отношение расстояние между траекториями в конце ${i}$ - го
шага к начальному расстоянию.

\newpage
