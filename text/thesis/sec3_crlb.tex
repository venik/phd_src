\subsection{Неравенство Крамера-Рао в задаче оценки параметров гармонического сигнала}

В разделе \ref{s2:crlb} было рассмотрено неравенство Крамера-Рао для оценки нижней границы среднеквадратической ошибки
при использовании любых оценок параметра.
В данном разделе будет приведено данное неравенство в задаче оценки среднеквадратической ошибки оценки частоты
гармонического сигнала. Приведенный вывод был взят из \cite{skon-clrb-report, rife-crlb-article}.

Пусть сигнал представлен как
\begin{center}
\begin{equation}
	\label{eq:crlb3_signal}
	Y(m) = S(m) + W(m) ,
	S(m) = A\cos(\omega_{c}m + \phi(m)),
	W(m) = n(m)
\end{equation}
\end{center}
${A}$ - амплитуда, ${\omega_c}$ - частота несущей сигнала, ${\phi(m)}$ - фаза несущей сигнала, ${n(m)}$ - аддитивный белый гауссов шум
с дисперсией ${\sigma_R^2}$, ${m}$ - индекс времени.

Так как ${W(m) = Y(m) - S(m)}$, функция плотности вероятности может быть записана как
\begin{center}
\begin{equation}
	\label{eq:crlb3_gauss}
	p({\bf Y}|{\bf S}) = \frac{1}{2\pi^{N/2}\sigma_R^N}
		exp\left[ -\frac{1}{2} \sum \limits_{m=1}^N(Y(m) - S(m))^2 / \sigma_R^2 \right]
\end{equation}
\end{center}

Тогда информационную матрицу Фишера можно записать как
\begin{center}
\begin{equation}
	\label{eq:crlb3_fisher_matrix}
	E \left[ \left( \frac{d \ln L({\bf Y}|{\bf S})}{d {\bf \alpha}} \right)\right]
\end{equation}
\end{center}
где ${\ln L({\bf Y}|{\bf S})}$ - логарифм функции правдоподобия, а вектор ${{\bf \alpha}}$ определяется как
\begin{center}
\begin{eqnarray}
	\label{eq:crlb3_alpha}
		\alpha =
		\left[ \begin{array}{c}
		A \nonumber	\\
		f 		\\
		\phi		\\
		\end{array} \right]
\end{eqnarray}
\end{center}

Логарифм функции правдоподобия может быть получен из функции плотности вероятности \ref{eq:crlb3_gauss}. Без учета констант он равен:
\begin{center}
\begin{eqnarray}
	\label{eq:crlb3_likehood}
	\ln L({\bf Y}|{\bf S}) = -\frac{1}{2} \sum \limits_{m=1}^{N}(Y(m) - S(m))^2 / \sigma_R^2
\end{eqnarray}
\end{center}

Производная логарифма функции правдоподобия с учетом параметров гармонического сигнала ${\alpha}$:
\begin{center}
\begin{eqnarray}
	\label{eq:crlb3_likehood_derivative}
	\frac{d \ln L({\bf Y}|{\bf S})}{d {\bf \alpha}} = \sum \limits_{m=1}^{N}(Y(m) - S(m)) \frac{d S(m)}{d \alpha} \frac{1}{\sigma_R^2}
\end{eqnarray}
\end{center}

После возведения в квадрат и взятия математического ожидания от \ref{eq:crlb3_likehood_derivative}:
\begin{center}
\begin{eqnarray}
	%\label{eq:crlb3_}
	E \left[ \left( \frac{d \ln L({\bf Y}|{\bf S})}{d {\bf \alpha}} \right)^2 \right]  =  \\
	 = E \left[ \sum \limits_{m=1}^{N}\sum \limits_{k=1}^{N} \frac{(Y(m) - S(m))(Y(k) - S(k))}{\sigma_R^4} \frac{d S(m)}{d \alpha} \frac{d S(k)^T}{d \alpha} \right]
\end{eqnarray}
\end{center}

Учитывая, что шум белый ${E \left[ (Y(m) - S(m))(Y(k) - S(k)) \right]=0}$ для всех значений, кроме ${i=k}$, где значение равно ${\sigma_R^2}$.
Таким образом
\begin{center}
\begin{eqnarray}
	%\label{eq:crlb3_}
	E \left[  \frac{d \ln L({\bf Y}|{\bf S})}{d {\bf \alpha}} \right]^2 =
	\sum \limits_{m=1}^{N} \frac{1}{\sigma_R^2} \frac{d S(m)}{d \alpha} \frac{d S(m)^T}{d \alpha}
\end{eqnarray}
\end{center}
где
\begin{center}
\begin{eqnarray}
	%\label{eq:crlb3_}
		\frac{d S(m)}{d \alpha} = 
		\left[ \begin{array}{c}
		\frac{\partial S(m)}{\partial A} \nonumber	\\
		\frac{\partial S(m)}{\partial f} 		\\
		\frac{\partial S(m)}{\partial \phi} \nonumber - 	\\
		\end{array} \right]
\end{eqnarray}
\end{center}
Информационная матрица Фишера может быть записана как
\begin{center}
\begin{eqnarray}
	\label{eq:crlb3_alpha}
		\left[ \begin{array}{ccc}
		B_{11} & B_{12} & B_{13}  \\
		B_{21} & B_{22} & B_{23}  \\
		B_{31} & B_{32} & B_{33}  \\
		\end{array} \right]
\end{eqnarray}
\end{center}
где
\begin{center}
\begin{eqnarray}
	%\label{eq:crlb3_}
	B_{11} & = & \sum \limits_{m=1}^N \frac{1}{\sigma_R^2}(\sin(\omega_c(m)+\phi))^2 \\
	B_{22} & = & \sum \limits_{m=1}^N \frac{1}{\sigma_R^2}(2\pi \Delta m A \cos(\omega_c(m)+\phi))^2 \\
	B_{11} & = & \sum \limits_{m=1}^N \frac{1}{\sigma_R^2}(A\cos(\omega_c(m)+\phi))^2 \\
	B_{12} & = & B_{21} = \sum \limits_{m=1}^N \frac{1}{\sigma_R^2}(2\pi \Delta m A\sin(\omega_c(m)+\phi)\cos(\omega_c(m)+\phi)) \\
	B_{13} & = & B_{31} = \sum \limits_{m=1}^N \frac{1}{\sigma_R^2}(2\pi \Delta m A\sin(\omega_c(m)+\phi)\cos(\omega_c(m)+\phi)) \\
\end{eqnarray}
\end{center}

Матрица обратная матрице ${\bf B}$ содержит границы неравенств Крамера-Рао для каждого неизвестного элемента вектора ${\alpha}$. Например
оценка частоты удовлетворяет неравенству ${E[\omega_c - \hat{\omega}_c] \ge \sqrt{B^{22}}}$.

Используя данные оценки можно получить базу оценки, которую улучшить невозможно.
