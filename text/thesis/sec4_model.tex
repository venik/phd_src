\subscription{Модель сигнала}

Для верификации развиваемых в работе подходов можно использовать данные, полученные при захвате в режиме онлайн или же
специально сгенерированные данные с использованием аппаратной или программной платформы. Преимуществом второго подхода
является то, что мы изначально знаем количество источников сигнала и их параметры (частоту, фазу ПСП, количество отраженных лучей).
В данной работе для верификации предложенных подходов будет использоваться второй пождход.

В разделе 1 уже рассматривались источники помех в системе с расширенным спектром Navstar GPS:
\begin{itemize}
	\item {данные эфемерид - ошибки в позициях спутников;}
	\item {часы спутника - ошибки в переданных данных о времени;}
	\item {ионосфера - ошибки в коррекции псевдодальностей обусловленных ионосферными эффектами;}
	\item {тропосфера - ошибки в коррекции псевдодальностей обусловленных обусловленных тропосферными эффектами;}
	\item {многолучевость - эффект отраженных сигналов принятых антенной;}
	\item {приемник - ошибки в измерениях обусловленные термальным шумом, точностью ПО и т.д.}
\end{itemize}

Генераторы сигнала можно разделить на программные и аппаратные. Программные генераторы позволяют сгенерировать
ВЧ сигнал с заранее заданными параметрами. В качестве аппаратных платформ для генерации сигнала можно использовать, например,
NI GPS Simulator. Данное решение позволяет сгенерировать практически произвольный сигнал за
счет большого количества настраиваемых параметров. Вместе с тем, аппаратные решения обладают существенным недостатком - высокая цена.

В данной работе будет использован программный подход к генерации сигнеала с расширенным спектром. В качестве модели была выбрана модель
сигнала с расширенным спектром СНС Navstar GPS. Существует много программных моделей данного сигнала,
например \cite{hannah_phd, burns_model, corbell_model, crs_model, brown_model}.

\subsection{Схема эксперимента}
Для проверки развиваемых в данной работе подходов построена имитационная модель системы передачи данных с ШПС.
Модель сигнала представлена в выражении \ref{eq:gps_signal}. Так как в работе развивается два подхода для сигнала
с АБГШ и с интерференционной помехой. Имитационная модель должна позволять выбрать необходимое количество источников сигнала.
Для проверки усовершенствованного алгоритма вычисления АКФ для компенсации окрашенной и белой аддитивной шумовой помехи
имитационная модель должна позволять добавлять АБГШ к генерируемому сигналу. 

Функциональная схема системы передачи информации представлена на рисунке \ref{pic:XXX}. Как уже было отмечено, бит данныx ${D_k(t)}$
принят за констату, так при ДФМ переход нарушает гармоническую структуру входного сигнала и детектирование становится невозможным,
она учтена в модели как неизвестная начальная фаза. Несущая сигнала модулируется заданной ПСП с периодом 1023 и длительностью 1 мкс.
Частота сигнала смещена на от центральной частоты для моделирования Допплеровского смещения. Рассолгасование и нестабильность
осцилляторов на стороне передатчика учтено в допплеровском смещении. Влияние этого рассогласования крайне невилико в сравнении
со смещением частоты, обусловленным допплеровским смещением в следствии движения передающего и принимающего сегментов.

К полученному сигналу добавляется
сигнал, модулированный другой ПСП при необходимости моделирования интерференционной помехи и АБГШ. Полученная смесь подавалась
на алгоритм детектирования для определения рабочих характерстик развиваемых в работе подходов.
