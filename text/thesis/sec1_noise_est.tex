\subsection{Постановка задачи оценки шума}
\label{sssec:sec1_noise_est}

Задача оценки отношения сигнал-шума (ОСШ) является одной из ключевых при детектировании сигналов.
ОСШ используется в задаче определения порога детектирования рисунок \ref{pic:sec1_gnss_system}.
При превышении порога принимается решение о наличии сигнала в принимаемой смеси, если же порог
не превышен, считается, что сигнал данного в смеси не присутствует.

Пусть для данной задачи входной сигнал описывается соотношением \cite{presti_ieee}:
\begin{center}
\begin{equation}
	\label{eq:noise_est_signal}
	s_C[n]=\sqrt{P_d}D[n] + \sqrt{P_n}\eta[n]
\end{equation}
\end{center}
где $D[n]$ - биты навигационного сообщения, $\eta[n]=\eta_{Re} + j\eta_{Im}$ - комплексный шум,
$P_d$ - мощность сигнала, а $P_n$ - мощность шума (обе величины берутся на выходе коррелятора).
Стоит отметить, что $D[n]=b_{n}e^{j\theta_n}$, где $b_n=\pm{1}$ для сигналов с двоичной модуляцией, а
$\theta_n$ - остаточная фазовая ошибка от контура ФАПЧ слежения за частотой.
Тогда ОСШ для $s_C[t]$ можно представить как:
\begin{center}
\begin{equation}
	\label{eq:noise_est_snr}
	R_e=\frac{P_d}{P_n}
\end{equation}
\end{center}
%От выражения \ref{eq:noise_est_snr} можно перейти к соотношению количества шума на герц $C/N_0$:
%\begin{center}
%\begin{equation}
%	\label{eq:noise_est_cn}
%	R_e=\frac{C}{N_{0}B_{eqn}}\Rightarrow\frac{C}{N_0}=R_e B_{eqn}
%\end{equation}
%\end{center}
%где ${B_{eqn}}$ - нормализованная эквивалентная шумовая полоса
%В \cite{presti_ieee} показано, что $B_{eqn}$ можно выразить:
%\begin{center}
%\begin{equation}
%	\label{eq:noise_est_beqn}
%	B_{eqn}=\frac{1}{T_{int}}
%\end{equation}
%\end{center}
%где ${T_{int}}$ - время интегрирования.

%%%%%%%%%%%%%%%%%%%%%%%%%%%%%%%%%%%%%%%%%%%%%%%%%%%%%%
\subsubsection{Алгоритм оценки ОСШ
\textquotedblleftдействительный сигнал - комплексный шум\textquotedblright}
\label{sssec:rscn}
В иностранной литературе он называется "Real Signal - Complex Noise (RSCN)". Введем перевод
"Действительный сигнал - комлексный шум (ДСКШ)".
Данный алгоритм рассмотрен в статьях \cite{badke_rscn, presti_insidegnss, presti_ieee}.

Если рассматривать идеально синхронизированный сигнал, тогда в синфазном контуре будет
находится сигнал и АБГШ, в то время как в квадратурном плече будет находится только шум,
независимый и одинаково распределенный с шумом в синфазном контуре. Данный факт может
быть использован для оценки ОСШ в программном приемнике:
\begin{center}
\begin{equation}
	\label{eq:rscn_noise_power}
	\hat{P_n} = \frac{2}{N}\sum^N_{v=1}|s_{C,Im}[v]|^2
\end{equation}
\end{center}
где ${s_{C,Im}[v]}$ - мнимая часть от ${s_C}$.

\begin{center}
\begin{equation}
	\label{eq:rscn_total_power}
	\hat{P}_{tot} = \frac{1}{N}\sum^N_{v=1}|s_{C}[v]|^2
\end{equation}
\end{center}
где ${P_{tot}}$ - оценка энергии смеси.

\begin{center}
\begin{equation}
	\label{eq:rscn_data_power}
	\hat{P_d} = \hat{P}_{tot} - \hat{P_n}
\end{equation}
\end{center}

\begin{center}
\begin{equation}
	\label{eq:rscn_snr}
	\hat{R_e} = \frac{\hat{P_d}}{\hat{P_n}} = \frac{\hat{P}_{tot} - \hat{P_n}}{\hat{P_n}} 
\end{equation}
\end{center}

Очевидно, что данный метод является чувствительным к сдвигу фазы, который приводит к переходу энергии
в квадратурный контур. Любой остаточный сдвиг фазы ведет к возрастанию энергии шума в квадратурном
контуре (это видно из \ref{eq:rscn_noise_power}).

%%%%%%%%%%%%%%%%%%%%%%%%%%%%%%%%%%%%%%%%%%%%%%%%%%%%%%
\subsubsection{Алгоритм Signal-to-Noise Variance}
\label{sssec:snv}

Данный алгоритм был представлен в \cite{snr_pauluzzi, snr_li}. Квадратичный ОСШ оценщик, основан на 1
и 2 моменте семплов сигнала:

\begin{center}
\begin{equation}
	%\label{eq:rscn_data_power}
	\hat{P_{d}} = (\frac{1}{N} \sum \limits_{v=1}^N \left| s_{C,Re}[v] \right|)^2
\end{equation}
\end{center}
где ${s_{C,Re}[v]}$ - действительная часть от ${s_C}$.

\begin{center}
\begin{equation}
	%\label{eq:rscn_data_power}
	\hat P_{tot} = \frac{1}{N} \sum \limits_{v=1}^{N} \left|s_C[v] \right| ^2
\end{equation}
\end{center}

\begin{center}
\begin{equation}
	%\label{eq:rscn_snr}
	\hat{R_e} = \frac{\hat P_d}{\hat P_{tot} - \hat P_d}
\end{equation}
\end{center}

%%%%%%%%%%%%%%%%%%%%%%%%%%%%%%%%%%%%%%%%%%%%%%%%%%%%%%
\subsubsection{Алгоритм Beaulieu}
\label{sssec:beaulieu}

Данный алгоритм был представлен в статье \cite{snr_beaulieu}.
\begin{center}
\begin{equation}
	%\label{eq:rscn_data_power}
	\hat{P_{n,v}} = (\left| r_{C,Re}[v] \right| - \left| r_{C,Re}[v-1] \right|)^2
\end{equation}
\end{center}
где случайная величина ${P_{n,v}}$ - мгновенное значение энергии шумовой компоненты ${\eta[v]}$.

\begin{center}
\begin{equation}
	%\label{eq:rscn_data_power}
	\hat{P_{d,v}} = \frac{1}{2}(r_{C,Re}[v]^2 + r_{C,Re}[v-1]^2)
\end{equation}
\end{center}
где случайная величина ${P_{d,v}}$ - мгновенное значение энергии сигнала ${s_C[v]}$.

\begin{center}
\begin{equation}
	%\label{eq:rscn_snr}
	\hat{R_e} = \left[ \frac{1}{N} \sum \limits_{v=1}^{N} \frac{\hat P_{n,v}}{\hat P_{d,v}} \right]^{-1}
\end{equation}
\end{center}

%%%%%%%%%%%%%%%%%%%%%%%%%%%%%%%%%%%%%%%%%%%%%%%%%%%%%%
\subsubsection{Алгоритм основанный на методе моментов}
\label{sssec:mm}
Данный алгоритм был представлен в \cite{snr_pauluzzi}. Он использует 2 и 4 статистические моменты для раздельной оценки
мощности сигнала и шума.
\begin{center}
\begin{equation}
	%\label{eq:rscn_data_power}
	\hat M_2 = \frac{1}{N} \sum \limits_{v=1}^{N} \left|s_C[v] \right| ^2
\end{equation}
\end{center}
\begin{center}
\begin{equation}
	%\label{eq:rscn_data_power}
	\hat M_4 = \frac{1}{N} \sum \limits_{v=1}^{N} \left|s_C[v] \right| ^4
\end{equation}
\end{center}
где ${M_2}$ и ${M_4}$ - 2 и 4 моменты случайной величины.

\begin{center}
\begin{equation}
	%\label{eq:rscn_data_power}
	\hat P_d(M_2, M_4) = \sqrt{2 \hat M^2_2 - \hat M_4} 
\end{equation}
\end{center}

\begin{center}
\begin{equation}
	%\label{eq:rscn_data_power}
	\hat P_n(M_2, M_4) = \hat M_2 - \hat P_d
\end{equation}
\end{center}

\begin{center}
\begin{equation}
	%\label{eq:rscn_snr}
	\hat{R_e} = \frac{\hat{P_d}(\hat{M_2}, \hat{M_4})}{\hat P_n(\hat{M_2}, \hat{M_4})}
\end{equation}
\end{center}

%%%%%%%%%%%%%%%%%%%%%%%%%%%%%%%%%%%%%%%%%%%%%%%%%%%%%%
%\subsubsection{Алгоритм Narrowband-Wideband Power Ratio}
%\label{sssec:nwpr}
%
%Данный алгоритм был представлен в \cite{parkinson_1996}, а так же в нескольких других книгах по СНРС GPS. Особенностью данного алгоритма
%является то, что это единственный алгоритм производящий оценку ${C/N_0}$, а не ОСШ, который может быть преобразован ${C/N_0}$.
%
%\begin{center}
%\begin{equation}
%	%\label{eq:rscn_data_power}
%	WBP_k = \sum \limits_{m=1}^{M} \left|s_C[kM+m] \right| ^2, k=0,1,...(\frac{N}{M}-1)
%\end{equation}
%\end{center}
%
%\begin{center}
%\begin{equation}
%	%\label{eq:rscn_data_power}
%	NBP_k = \left(\sum \limits_{m=1}^{M} \left|s_{C,Re}[kM+m] \right| \right)^2 + \left(\sum \limits_{m=1}^{M} \left|r_{C,Im}[kM+m] \right| \right)^2
%\end{equation}
%\end{center}
%
%\begin{center}
%\begin{equation}
%	%\label{eq:rscn_data_power}
%	\hat \mu_{NP} = \frac{M}{N} \sum \limits_{k=0}^{N/M-1} \frac{NBP_k}{WBP_k}
%\end{equation}
%\end{center}
%
%\begin{center}
%\begin{equation}
%	%\label{eq:rscn_data_power}
%	\gamma = \frac{C}{N_0} = \frac{1}{T_{int}} \frac{\hat \mu_{NP} - 1}{M - \hat \mu_{NP}}
%\end{equation}
%\end{center}

%%%%%%%%%%%%%%%%%%%%%%%%%%%%%%%%%%%%%%%%%%%%%%%%%%%%%%
\subsubsection{Выводы}
Приведенные алгоритмы подробно рассмотрены в \cite{presti_ieee}. Получены их оценки по количеству операций, а так же
исследованы свойства аппроксимации данных функций. Следует отметить, что данные алгоритмы работают только с синхронизированным сигналом.
В виду этого является очень важным получить достаточно точную оценку частоты на ранних стадиях детектирования сигнала

\newpage
