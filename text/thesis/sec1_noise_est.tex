\subsection{Постановка задачи оценки шума}
\label{sec1_noise_est}

Задача оценки отношения сигнал-шума (ОСШ) является одной из ключевых при детектировании сигналов.
ОСШ используется в задаче определения порога детектирования рисунок \ref{pic:sec1_gnss_system}.
При превышении порога спутник считается задетектированным, если же порог
не превышен, считается, что сигнал данного спутника в данных не присутствует.

Пусть для данной задачи входной сигнал описывается соотношением \cite{presti_ieee}:
\begin{center}
\begin{equation}
	\label{eq:noise_est_signal}
	s_C[t]=\sqrt{A_d}D[n] + \sqrt{A_n}\eta[n]
\end{equation}
\end{center}
где $D[n]$ - биты навигационного сообщения, $\eta[n]=\eta_{Re} + j\eta_{Im}$ - комплексный шум,
$P_d$ - мощность сигнала, а $P_n$ - мощность шума (обе величины берутся на выходе коррелятора).
Стоит отметить, что $D[n]=a_{n}e^{j\theta_n}$, где $a_n=\pm{1}$ для сигналов с двоичной модуляцией, а
$\theta_n$ - остаточная фазовая ошибка от контура ФАПЧ слежения за частотой.
Тогда ОСШ для $s_C[t]$ можно представить как:
\begin{center}
\begin{equation}
	\label{eq:noise_est_snr}
	\lambda_C=\frac{P_d}{P_n}
\end{equation}
\end{center}
От выражения \ref{eq:noise_est_snr} можно перейти к соотношению количества шума на герц $C/N_0$:
\begin{center}
\begin{equation}
	\label{eq:noise_est_cn}
	\lambda_C=\frac{C}{N_{0}B_{eqn}}\Rightarrow\frac{C}{N_0}=\lambda_{C}B_{eqn}
\end{equation}
\end{center}
В \cite{presti_ieee} показано, что $B_{eqn}$ можно выразить:
\begin{center}
\begin{equation}
	\label{eq:noise_est_beqn}
	B_{eqn}=\frac{1}{T_{int}}
\end{equation}
\end{center}
где ${T_{int}}$ - время интегрирования.


\newpage
